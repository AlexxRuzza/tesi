
%\usepackage{fullpage}
%\usepackage{amsmath}
%\usepackage{amsthm}
%\usepackage{amsfonts}
%\usepackage{tikz-cd}
%\usepackage{chngcntr}
%\usepackage[utf8x]{inputenc}
%\usepackage{enumerate}

\begin{titlepage}
	
	\begin{figure}
		\centering
		\includegraphics[width=424pt]{copertina/logo.jpg}%A QUESTO LINK TROVATE I MARCHI PER LA TESI AGGIORNATI E DIVISI PER FACOLTà: http://www.unimi.it/ateneo/37094.htm
		\vspace{0.5 cm}
	\end{figure}
	
%DATO CHE NEI MARCHI SONO GIà PRESENTI SIA IL NOME DELL'UNIVERSITà SIA QUELLO DELLA FACOLTà, NON VANNO RISCRITTI. IN OGNI CASO AGGIUNGO IN COMMENTO COME SAREBBERO:	
%\begin{center}
%	{\Huge \textsc{Università degli Studi di Milano} }\\
%\end{center}
%\begin{center}
%{\Huge Facoltà XXXXX}\\
%\end{center}

\begin{center}
{\LARGE Corso di Laurea Triennale in Fisica}
\end{center}

\begin{center}
\vspace{3 cm}
{\Large \textsc{Design of a machine learning model for the characterisation of young planets from dust morphologies in discs} }
\end{center}
\par
  \vspace{3 cm}
  
  \begin{flushleft}
  		 Relatore:\\ Prof. Giuseppe Lodato\\
		 
  		 \noindent Correlatore:\\ Dott. Stefano Carrazza
  \end{flushleft}
  \vspace{1 cm}
  \begin{flushright}
  	Tesi di Laurea di:\\ Alessandro Ruzza\\ Matricola: 931750
  \end{flushright}
    	  
\vfill
\begin{center}
	{\large Anno Accademico 2020/2021}
\end{center}
\end{titlepage}
