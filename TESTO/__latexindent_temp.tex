\documentclass[a4paper,10pt]{report}
\usepackage[utf8]{inputenc}
\usepackage{verbatim}
\usepackage{physics}
\usepackage{amsmath}
\usepackage{amssymb}

%opening
\title{Design of a machine learning model for the detection of young planets}
\author{}

\begin{document}

\maketitle

\addcontentsline{toc}{chapter}{Abstract}
\begin{abstract}
    Protoplanetary discs present substructures,
    such as axisymmetric regions of luminosity depletion (gaps),
    that can be explained by the presence of forming planets. 
    Features of these objects can be inferred from their
    observation and analysis. A remarkable example is the 
    estimation of the planetary mass from the gaps morphology.
    The approaches currently used, empirical formulae or numerical
    simulations, are both limited in precision or applicability.
    In this thesis we propose a machine learning approach:
    using a neural network to infer this information from 
    disk images with the requirement of the least amount
    of physical features not directly observable.
    Possible future developments of such models require 
    data for the train and test phases.
    We design and build a database for this purpose collecting 
    data obtained from numerical simulations and providing an 
    easy-to-use interface for the implementation of machine learning
    models using TensorFlow libraries.
\end{abstract}

\tableofcontents

\chapter{Introduction}


\chapter{Protoplanetary discs}
%Very brief introduction (1/2 sentences) about what a protoplanetary disc is,
%how they generate, where we can find them and why they are studied.

During its formation, a star accrete its mass drawing matter from the surrounding structures
of gas and dust. This matter is usually organized in discs which evolve along with the star.
At some point in the star formation process, after $\sim 10^5$ yr, when most of the disc matter has accreted into the star the 
remaining disc is called protoplanetary disc. At this stage the disc temperature and emission is mainly due to the 
irradiation it receives from the star.
In these astronomical objects planet formation takes place: solid fragments called planetesimals form from disc matter and start to accrete 
their mass with different mechanisms.
The study of protoplanetary discs is thus of primary importance in the detection and characterization of young planets 
allowing the development and test of planet formation theories.

\section{Structural properties}

\begin{comment}
Here I am going to discuss some key properties of protoplanetary discs and give
gross estimates of their typical values. I am going to discuss: 

— what discs are made of

— absolute and relative masses of gas and dust components

— dimension and distance

— their age

— temperature
\end{comment}

Protoplanetary discs form in the context of star formation. This process takes place in specific regions
known as "star-forming regions" filled with the interstellar medium: a mixture of gases, mainly hydrogen and helium, enriched with some
heavier elements. The gravitational collapse of denser regions gives birth to stars and eventually, after $\sim 10^5$ yr, leads the formation of protoplanetary discs.
These regions had been widely studied and classified. Many features of protoplanetary discs are intimately linked to the 
star formation process and to the environment they generate from.

The disc structure appears as simply a consequence of angular momentum conservation. We will always consider axisymmetric discs. 
In the following sections we will use cylindrical coordinates 
defining the frame of reference in figure 1, to discuss disc properties. 

%TODO: insert image with the frame of reference of the disc

Most of the discs were observed at distances of about 150 pc with a typical diameter of 100 a.u. meaning that they span approximately
1 arcsec of the sky. 

Two main components can be distinguished according to their physical state: gas and solids.
The solid component consists in dust and debris of different dimensions, going from micrometers to few 
meters, which build up about the 1\% of the total disc mass. Despite being a fraction of the disc the solid components
are actually the easier to observe and measure due to some features which will be further discussed in section {}

Dust and solid fragments are embedded in a gaseous medium which provide most of the disc mass. The most abundant molecule is $H_2$ which is challenging to observe due to
its lack of a dipole moment. Measures related to less abundant molecules, such as HD or CO, provide insights into the properties of the gas component. 

The overall mass of protoplanetary discs is measured to account for some jupyter masses. Estimates of these quantities can provide upper limits
to the masses of forming planets.

Gas and dust temperature is strictly related to sundry factors both in the dynamic and radiative emission.
Its value changes with the radial and vertical distance to the star going from hundreds of K to approximately 20K.
The interstellar medium, the background of observations, has observable features compatible with a temperature of about 10K.


\section{Disc dynamics and evolution}

\begin{comment}
Here I am going to explain how the dynamic of gas and dust is modelled.

I am going to provide the equations describing the vertical structure, explain
the meaning of the aspect ratio and the viscous forces at play.

I will also explain the model describing the interaction
between the gas and solid components (Epstein force, stokes number).

Finally, I am going to cite other forces and effects which play a role in disc dynamics,
such as magnetorotational instability, turbulence, winds, photoevaporation, ...
\end{comment}

The disc evolution is governed and properly described by fluidodynamics. (+forze gravità)
%add NS equations (?) 
Some assumptions need to be made in order to acquire a predictive model of practical use.
The first one is called ``thin-disc approximation'' which consists in assuming that the radial distance is greater than 
the vertical typical length scale $H$, thus requiring $H/R \ll 1$. This quantity, called aspect-ratio, has been measured 
showing a value of $\sim 0.1$ which justify the approximation. This assumption allows the study of disc properties integrating
the equations along the vertical direction
Self-gravitation of the disc will be neglected. The stability condition of the disc against self-gravity can be written in the form

\begin{equation}
    \frac{M_{disk}}{M_{star}} \lesssim  \frac{H}{R}
\end{equation}

This condition is well satisfied in the late epochs when protoplanetary discs are studied.

Keeping in mind these assumptions, I am going to further describe how the fluidodynamic description is applied to model the disc structure and  
dynamics. First, I am going to focus on the gas component which is modelled as an ideal gas.
The results obtained in the study of the gas account for most of the macroscopical disc features due to its
relative abundance with respect to the solid component.

The vertical structure of the gas is determined by a steady-state solution of the hydrodynamical equations and the Poisson equation that
accounts for the gravitational potential. The assumptions stated above allow a great simplification of this problem leading to the vertical density profile
\begin{equation}
    %TODO: insert vertical profile
\end{equation}
The equation above offers a quantitative definition of the aspect-ratio $H$, previously introduced as the typical height length scale: it is the 
standard deviation of the Gaussian describing the vertical density profile. It also provides the relation

\begin{equation}
    H = \frac{c_s}{\Omega_K}
\end{equation}

Where $c_s$ is the sound speed defined as $c_s^2 = \dv{P}{\rho} = \frac{k_BT}{\mu m_p}$ while $\Omega_K$ is the Keplerian 
angular velocity.

In the simplest disc models shear viscosity is taken into account with the introduction of the new parameter $\alpha$
called ``Shakura-Seneyev viscosity'' which gathers all the ignorance about viscous processes.
This parameter is introduced through the following reasoning.
In an ideal gas $\nu = \frac{1}{3}c_s\lambda$, with $\lambda$ indicating the mean free path of particles in the fluid.
In protoplanetary discs we have to consider also the turbulent regime. Dimensional arguments lead to the
assumption of $\nu_T \sim v_T \lambda_T$. In this equation $v_T$ indicates a typical velocity of turbulent motions which should 
satisfy $v_T \lesssim c_s$, upper velocities would lead to shocks thermalizing the turbulent motion. The $\lambda_T$ factor 
represents the typical length scale in the turbulent regime which, assuming isotropic turbulence, can not be greater than $H$, the disc heigth.

The dust component is modelled as a pressure less fluid with grains of different dimensions coupled with the gas medium. The strength of the coupling is expressed by the Stokes number $St$.
Two drag forces come into play depending on the grain size. If $s \lesssim \lambda$ (with $\lambda$ indicating the mean free path of molecules within the disk), the drag force is called Epstein drag. 
In this regime, which is usually the most relevant for most particle sizes, the drag is caused by the difference in the frequency of 
collisions with the gas molecules between the front and back size of the grain as a consequence of its motion in the gas medium.
Once particles reach sizes much larger than the molecular mean free path they begin to experience a force of different nature called Stokes drag.




\section{Observations}
Here I am going to explain how discs are observed presenting the different observational primers 
for the gas and the dust component. I am going to present some links between structural and observational 
properties (such as $\lambda \sim s/2\pi$).

I am also going to explain which is the best image resolution currently achievable.

\section{Planet formation}
Here I am going to discuss how planets are formed within these discs. 
I will explain: 

    — how they accrete their mass

    — the forces they experience and thus the radial drift

    — the substructures they form in the disc

    — typical mass values and their relation with substructures (qualitatively) 

\section{Gaps}
Here I am going to:

— explain how planets are not the only thing that generates gaps

— present the radial profile of a gap in dust and gas densities

— explain the differences between gap structures in gas and dust 

— explain how depth and width are defined


\chapter{State-of-the-art investigative techniques}


\section{Addressed questions}

Here I am going to explain which features are commonly extrapolated 
from discs images and why they are important.
Then I am going to state that in the following paragraphs I will mainly focus 
on methods for the determination of embedded planets' masses.

\section{Empirical formulae}

In this section I am going to explain that many disc features can be analytically 
linked using simple linear or power laws.

\subsection{Planet mass and gap width}
I am going to focus on the link between planet's mass and gap width,
providing the `Lodato' and `Kanagawa' models.

\subsection{Strengths and limitations}
Here I am going to discuss the strengths and limitations of the analytical approach.
From this section it should be clear why numerical simulations are preferred.

\section{Numerical approach}
In this section I am going to explain how disc features can be inferred 
from simulations of the entire disk. I am going to present the possibilities
in the choices of the simulating software.
I am then going to focus on a specific choice and discuss the main steps of the 
simulation workflow.

This section plays a double purpose: it presents the current approach for
the study of protoplanetary discs and explains how the images used to build the 
database were generated.

\subsection{Hydrodynamical simulations}

Here I am going to give some background about phantom and the type of 
data it generates.

\subsection{Radiative transfer}

Here I am going to discuss the software used for radiative transfer: MCFOST.
I will explain why this step is performed and the meaning of the results obtained.

\subsection{Generation of synthetic images}

Here I am going to discuss the different methods that can be used to simulate the limitations 
of observing instruments. I will further explore some key features of pymcfost.

\subsection{Strengths and limitations}

Here I am going to discuss the strengths and limitations of numerical simulations in the context of 
disc analysis. From this subsection the need for a faster method should emerge.

\chapter{Machine learning and neural networks}

Very brief introduction to machine learning. (Birth and definition)

\section{Neural networks}

Basic idea behind neural networks.

\subsection{The perceptron}

Here I am going to present the perceptron model explaining:

— how outputs are generated from inputs

— what are the weights tuned during the training process

— what is the activation function

\subsection{Architecture and types}

In this subsection I am going to explain how perceptrons are organized 
within a neural network. The concept of layer (and hidden layer) will be explained.
After acknowledging the existence of many types of neural networks
I am going to focus on the Feedforward model.

Then I am going to discuss the possibility to improve a 
feedforward neural network with convolutional layers.
I will explain how they work, what they are designed for and how they can be exploited.

\subsection{Training}

Here I am going to explain the key steps of the training process.
I will explain how it works, what algorithm can be used and the concepts
of loss functions and metrics.

\section{Strengths and limitations}

In this section I am going to discuss the strengths and limitations 
of machine learning techniques, focusing on aspects with direct relevance
to this thesis.

\subsection{The universal approximation theorem}

Here I am going to discuss the flexibility of neural networks and the
theoretical framework that proves their potential.

\subsection{Hyperparameters}

Here I am going to write about hyperparameters. I am going to list them
providing basics explanations about how their value can affect the model.
They will be presented as both a strength and a limitation.

This subsection should highlight the importance of carefully tuning the hyperparameters.

I am going to cite the existence of algorithm for doing this job automatically and more
efficiently than by simple trial and error.

\subsection{Training data}

Here I am going to discuss the importance of having a large dataset in the implementation
of a machine learning model. I am going to weight pro and cons of this data driven 
approach.

\subsection{Overlearning and underlearning}

Here I am going to discuss overlearning and underlearning. 
I am going to:

    — define them

    — explain their causes

    — provide a method for their  detection 

    — discuss the solutions (early stopping, vary the number of trainable parameters, ...)

\subsection{Computational complexity}
Here I am going to discuss the computational complexity of machine learning
 algorithms in comparison with numerical simulations.

I have to highlight that the most resource requiring part is the training process.
The aim is thus to obtain a trained neural network that can be deployed and used for the study of
a wide range of different disc images without the need to re-train it.

\section{Machine learning and protoplanetary discs}

In this section I am going to develop the idea of applying machine learning methods to the  study of protoplanetary
discs.

\subsection{The proposed approach}

Here I am going to discuss the approach we want to propose.
 I am going to give some details about the 
suggested architecture for the neural network and what 
we expect to be able to predict with the trained
model.

I am going to think about possible scenarios which can take advantage from this approach
(ex. large surveys with lots of disc images: the neural network 
could quickly provide measures of their physical properties (?) )

\subsection{Previous attempts in literature}

Here I am going to discuss the Sayantan Auddy and Min-Kai Lin's paper citing their results
and explaining the main differences with the approach we propose.

\chapter{Dataset design}

In this chapter I am going to unfold the main part of my work: the design and implementation of the dataset.
Here I am going to recall the key features of a good dataset for machine learning.

\section{The data}

In this section I am going to present what the dataset is made of: fits images and the list of parameters included in the fits files 
and in data.js. I am going to briefly explain why they were included and their possible use.

I am also going to present images showing the data distribution over the parameters space, discussing
 which of them are properly explored and which not.

\section{Structure and interface}

Here I am going to discuss the structure we designed for the db and the interface provided to 
access the data.

\section{Supporting scripts}

Here I am going to write about the scripts I wrote which allow the user to handle the database
and preprocess the results coming from MCFOST simulations.

\section{Expanding the dataset}

Here I will explain how the tools provided allow to easily expand the dataset. 
Then I am going to give future perspectives on how the database could be improved.

\chapter{Proof of concept}

Here I am going to present an example of model trained to predict planet's mass from 
disc images.
From the different models I tested (and I will test)
I am going to report here the one which gives the best result.
I am also going to cite the python libraries used to implement this model (TensorFlow).

\section{Adopted model}

Here I am going to explain the model used: number of layers, type of neural network, activation functions,
optimizer, metrics.

\section{Data pre-processing}

Here I will explain how I chose the data used during the training process.
This section should highlight the versatility of the dataset showing the possibility to split the 
data according to our needs.

\section{Results}

Here I will discuss the results obtained.

\chapter{Conclusions}
\section{Conclusion}
\section{Future perspectives}

\chapter*{Acknowledgements}
\addcontentsline{toc}{chapter}{Acknowledgements}

\chapter*{Bibliography}
\addcontentsline{toc}{chapter}{Bibliography}

\end{document}
