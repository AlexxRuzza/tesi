\documentclass[DIN, pagenumber=false, fontsize=11pt, parskip=half]{scrartcl}

\usepackage[utf8]{inputenc}
\usepackage[italian]{babel}
\usepackage[T1]{fontenc}
\usepackage[utf8]{inputenc}
\usepackage[T1]{fontenc}
\usepackage{textcomp}
\usepackage{systeme}
\usepackage{circuitikz}
\usepackage{amsmath}
\usepackage{epsf}
\usepackage{hyperref}
\usepackage{graphicx}  
\usepackage{epstopdf}
\usepackage{geometry}
\usepackage{color}
\usepackage{amssymb}
\usepackage{booktabs}
\usepackage{verbatim}
\usepackage{tikz-feynman}
\usepackage{physics}
\usepackage{siunitx}
% for matlab code
% bw = blackwhite - optimized for print, otherwise source is colored
\usepackage[framed,numbered,bw]{mcode}

% for other code

%\usepackage{listings}

\setlength{\parindent}{0em}

% set section in CM
\setkomafont{section}{\normalfont\bfseries\Large}

\newcommand{\mytitle}[1]{{\noindent\Large\textbf{#1}}}
\newcommand{\mysection}[1]{\textbf{\section*{#1}}}
\newcommand{\mysubsection}[2]{\romannumeral #1) #2}

\lstset{
    literate={~} {$\sim$}{1}
}

%===================================
\begin{document}

\sisetup{exponent-product = \cdot}
\noindent{Alessandro Ruzza, 931750} \hfill {Giugno 2021}\\

\mytitle{Raccolta informazioni}

\section{Grandezze fisiche}

\subsection{Idrodinamica}

\begin{center}
\begin{tabular}{l l | c c c c }
\toprule
       &              & valori tipici & phantom & mcfost & DB\\
\midrule       
 $M_P$ & \hyperref[mp]{planet mass} &  $(10-10^2) \text{M}_\oplus$  &  in & & \checkmark \\
 $M_*$ & star mass & $\sim 1\text{M}_\odot$& in & & \checkmark \\
 $M_g$ & gas mass & $\sim (1-10)\text{M}_j$ & in & & \checkmark \\
 $M_d$ & \hyperref[dustmass] {dust mass} & $\sim (10^{-3}-1)\text{M}_j$& comp & & \checkmark \\
 $\frac  HR$ & \hyperref[aspectratio]{aspect ratio} & (0.001)-0.1 & in (at $R_{ref}$) & & \checkmark \\
 $R_p$ & planet orbital radius & $(1-100)\text{au}$ & in & & \checkmark \\
 $R_{in} - R_{out}$ & simulation boundaries & $\sim (1-10^2)$au& in & \\
 $\alpha_{SS}$ & \hyperref[visc]{Shakura-Sunyaev viscosity} & $10^{-4}-10^{-2}$& \shortstack{in: desired,\\ out: min, max} & & \checkmark \\
 $\alpha_{AV}$ & \hyperref[artvisc]{artificial viscosity} &  & in &  & \\
 $St$ & \hyperref[stnum]{ Stokes' number} & $10^{-2} - 10^{-3}$& &&\checkmark\\
 $s$ & grain size & $(1-10^5)\mu$m& in & & \checkmark\\
 $\rho_g$ & grain density & $\sim \SI{1}{\g\per\cubic\cm}$& in & in &\checkmark \\
 $\epsilon$ & dust-gas ratio & 0.01 &  in\\
 $p$ & \hyperref[pind]{p index ($\Sigma_g$)} & 1 &in & &\checkmark \\
 $q$ & \hyperref[qind]{q index ($c_s$)} & 0.25 &in & & \checkmark \\
 $T$ & \hyperref[temp]{temperature} & $(20-100)\text{K}$& \shortstack{in (at \\ $R_{in}, R_{ref}, R_{out}$)}& comp&  \checkmark \\
 %$R^*_{accr}$ & star acccretion radius & & \\
 \bottomrule
\end{tabular}
\end{center}

\subsection{Trasferimento radiativo}
\begin{center}
\begin{tabular}{l l | c c c c}
\toprule
       &              & valori tipici & phantom & mcfost & DB\\
\midrule       

 $\kappa$ & opacity & & & in &\checkmark \\
$\tau$ & optical depth & & in & \\
$\beta$ & opacity spectral index & & & in &\checkmark \\
$T_*$ & star temperature & $\sim 4000$K & & in & \checkmark \\
\bottomrule
\end{tabular}
\end{center}

\subsection{Osservazioni}


\begin{center}
\begin{tabular}{l l | c c c c c}
\toprule
        &              & valori tipici & phantom & mcsoft & pymcfost & DB \\
     \midrule    
     $b_{maj}$ & FWHM beam (maj) & & & & in &\checkmark \\
     $b_{min}$ & FWHM beam (maj) & & & & in &\checkmark \\
     $b_{pa}$ & position angle beam (maj) & & & & in &\checkmark \\
     $d$ & distance & $\sim 150\text{pc}$ & & in & & \checkmark \\
     $\lambda$ & wavelength & & & &  &\checkmark\\
     $i$ & inclination &&in& in&&\checkmark \\
     $p.a.$ & position angle & & in &in & & \checkmark \\
\bottomrule
\end{tabular}
\end{center}

\subsection{Grandezze fisiche approfondite}

\subsubsection{Planet mass}
\label{mp}
%TODO: specificare quali sono i limiti e perchè
Per dischi con $St \ll 1$ la massa minima in grado di aprire un gap è \hyperref[source]{$^1$} 
\begin{equation}
 \frac{M_{min}}{M_*} = 0.3 \left(\frac HR \right)^3
\end{equation}

Per masse superiori alla massa termica $M_{th} = \left(\frac {H}{ R_p} \right)^3\cdot 3M_*$, massa per la quale il raggio di Hill è pari ad $H$, si possono presentare strutture più complesse: strutture a spirale, bordi del gap con vortici,...

\subsubsection{Dust mass}
\label{dustmass}
Viene ricavata dalle osservazioni nel continuo di emissione termica della polvere utilizzando la relazione
\begin{equation}
 I_\nu \propto \tau_\nu B_\nu(T) \propto \kappa_\nu \Sigma_s B_\nu(T)
\end{equation}
valida assumendo sia prevalente il limite otticamente sottile.

Misure ad una data lunghezza d'onda sono in pratica una misura della massa della componente solida di dimensioni comparabili con tale lunghezza d'onda.

Le misure a disposizione sono probabilmente valori sottostimati.

Nota: in alcuni casi potrebbe essere interessante il solo rapporto $M_P/M_*$ 
\subsubsection{Aspect ratio}
\label{aspectratio}
Definito come
\begin{equation}
h =H/R
\end{equation}
Imponendo la condizione di equilibrio idrostatico per il gas lungo z (ad R fissato) si ottiene che la densità assume una distribuzione
\begin{equation}
\rho_g = \rho_0 \cdot \exp(-\frac{\Omega_k^2z^2}{2c_s^2})
\end{equation}
che è una gaussiana con deviazione standard $H \equiv c_s/\Omega_k$

Sempre da questa relazione si ha anche $h = c_S/v_K$
Per un fluido perfetto $c_S=(\frac{k_BT}{\mu m_p})^{1/2}$, $H$ è quindi legato alla temperatura del disco

Legato anche alla stabilità/instabilità gravitazionale. Dall'espressione del parametro di Toomre, con alcune approssimazioni si ottiene la condizione per la stabilità
\begin{equation}
\frac{M_{disk}}{M_*} \gtrsim h
\end{equation}
Solitamente si assumono dischi sottili in cui $h \ll 1$, questa condizione impone anche una condizione sulle velocità: $c_s \ll v_\phi$.

Dalla componente radiale dell'equazione di Eulero si trova che la velocità angolare del gas presenta una correzione del secondo ordine in $h$ rispetto a quella Kepleriana
\begin{equation}
v_{\phi, gas} = v_K(1-qh^2)^{1/2}
\end{equation}

Misure che rivelano l'andamento della temperatura in funzione di R permettono di ottenere anche la curva di $h$.

\subsubsection{Stokes' number}
\label{stnum}
Il gas esercita sulle particelle di polvere una forza di diversa natura a seconda della dimensione della particella $s$

$s \lesssim \frac94 \lambda$, regime di Epstein $\bold{F_D} = -\frac{4\pi}3 \rho s^2 v_{th} \bold{v}$

$s \gtrsim \frac94 \lambda$, regime di Stokes $\bold{F_D} = -\frac{C_D}2 \pi s^2 \rho v \bold{v}$

Il numero di Stokes è definito come 
\begin{equation}
St = t_{stop}\Omega_K
\end{equation}

dove $t_{stop}$  è il tempo caratteristico di frenamento dovuto al drag viscoso in uno dei due regimi.

Per particelle $\lesssim 10\text{m}$ prevale il regime di epstein, nel quale
\begin{equation}
St = \frac\pi2 \frac{s\rho_g}{\Sigma_g}
\end{equation}
dove $\rho_g$  è la densità del materiale che costituisce la componente solida del disco

Il suo valore può variare con la posizione e anche con l'evoluzione del disco. 
Può essere fissato nelle simulazioni per non fissare la dimensione delle particelle solide. O viceversa può essere fissata la dimensione della polvere.

\subsubsection{Shakura-Sunyaev viscosity}
\label{visc}
Parametrizza la viscosità cinematica $\nu = \mu/\rho$ 

\begin{equation}
\nu = \alpha \cdot H \cdot c_s
\end{equation}
La viscosità influenza l'evoluzione radiale della densità superficiale $\Sigma_g$ 
\begin{equation}
\frac{\partial\Sigma}{\partial t}= \frac{3}{r} \frac{\partial}{\partial r} [r^{1/2}\frac{\partial}{\partial r}(\nu\Sigma r^{1/2})]
\end{equation}
da cui si ottiene una scala temporale legata alla viscosità
\begin{equation}
t_\nu \sim R^2/\nu
\end{equation}
Misure o stime:

viene stimato richiedendo quale deve essere il suo valore per riprodurre le scale di tempo tipiche dell'evoluzione del disco, come? analisi di indicatori di accrescimento:

- eccesso ultravioletto \\
- emissione H$\alpha$  dalla fotosfera della stella\\
- osservazioni nel (sub-)mm risolte spazialmente

Viene solitamente considerato costante in tutto il disco, potrebbe non esserlo ma l'utilità del modello risiede nel considerarlo costante

Si riescono ad ottenere prevalentemente limiti superiori, l'evoluzione radiale potrebbe infatti essere dovuta anche ad altri meccanismi

\subsubsection{Artificial Viscosity\hyperref[source]{$^2$}}
\label{artvisc}

Utilizzata per modellizzare la viscosità nelle simulazioni SPH.
Legata alla $\alpha_{SS}$ dalla relazione 
\begin{equation}
\alpha_{SS} \simeq \frac{\alpha_{AV}}{10}\frac{\ev{h_s}}{H}
\end{equation}

dove $\ev{h_s}$ è il valor medio azimutale (ad R fissato) della smoothing length delle particelle.
Per ottenere $\alpha_{SS}$ costante è necesario risolvere $H(R)$.

$\alpha_{AV}$ viene assunto costante in tutto il disco.

Vantaggi:
\begin{enumerate}
 \item[-] forza a considerare se $H$ è risolto o meno
 \item[-] è utile conoscere $\alpha_{AV}$ perchè $\alpha_{AV} \simeq 0.1$ rappresenta il limite inferiore sotto il quale la viscosità non è risolta dal metodo SPH.
\end{enumerate}

Svantaggi:
\begin{enumerate}
 \item[-] si ottengono alte viscosità di bulk $\frac 16 \alpha_{AV} c_s h$, tuttavia questo coefficiente è applicato a $\div v$ che per le simulazioni di dischi assume generalmente valori bassi.
\end{enumerate}


\subsubsection{Temperature}
\label{temp}
La temperatura del disco è determinata dal bilancio di calore proveniente principalmente da due sorgenti

- disco passivo: radiazione emessa dalla stella che viene intercettata dal disco \\($P=fL_*, \hspace{5pt} f\sim 1/4$) \\
- disco attivo: dissipazione dell'energia potenziale gravitazionale ($P \sim 
G \frac{M_*\dot{M}}{R_*}$)

La scelta del limite in cui trascurare una delle due sorgenti dipende principalmente dal rate di accrescimento $\dot{M}$  della stella. Dai valori ottenuti con le osservazioni sì può concludere che entrambi i processi possono essere rilevanti a seconda della situazione.

Fortemente legata alla forma del disco.

Per un disco piatto sottile (passivo): $T \propto R^{-3/4} \implies h \propto R^{1/8}$

Per un disco "flared" (passivo): $T \propto R^{-1/2} \implies h \propto R^{1/4}$ 

T(z) cresce con |z|

Temperature tipiche vanno da 10K, che è circa la temperatura del mezzo interstellare, a qualche 50/100K. Valori tipici sono 20/30K.

Per determinare la distribuzione di temperatura:

- simulare la distribuzione spettrale nell'infrarosso proponendo densità ed opacità del disco. Problema: degenerazioni del modello\\
- osservazioni in regime otticamente spesso

\subsubsection{p index}
\label{pind}
Indice della legge di potenza che descrive l'andamento della densità in funzione del raggio
\begin{equation}
    \Sigma_g(R) = \Sigma_{ref} \left( \frac{R}{R_{ref}}\right)^{-p}
\end{equation}

può essere attivata una modifica al profilo di densità chiamata ``exponential taper'' consistente in un termine moltipicativo che modifica la densità soprattutto ai bordi. Può assumere diverse forme a seconda di quella impostata.
Il profilo di densità utilizzato durante la simulazione viene scritto in *.discparams

Due tipi di exponential taper
\begin{align}
 \Sigma_g(R) = \Sigma_{ref} \left( \frac{R}{R_{ref}}\right)^{-p} &\cdot 
 \exp[-(R/R_c)^{(2-p)}] \\
 \Sigma_g(R) = \Sigma_{ref} \left( \frac{R}{R_{ref}}\right)^{-p} &\cdot
 [1-\exp(R-R_{out})]
\end{align}

viene usato il (14) nelle simulazioni di DSTau

\begin{center}
 \begin{tikzpicture}
  \draw[->] (-0.2, 0) -- (12, 0) node[above] {$R$};
  \draw[->] (0, -0.2) -- (0, 2) node[left] {$\Sigma_g$};
  \draw[domain=1:10, variable=\x, blue, smooth, thick] plot ({\x}, {2*\x^(-1)*2.718^(-(\x/7))});
  \draw[domain=1:10, variable=\x, red, smooth, thick] plot ({\x}, {2*\x^(-1)});
  \draw[red, thick] (7,2) -- (7.2, 2);
  \node at (9.5, 2) {without exponential taper};
  \draw[blue, thick] (7,1.5) -- (7.2, 1.5);
  \node at (9.25, 1.5) {with exponential taper};
 \end{tikzpicture}
\end{center}
Per calcolare $\Sigma_{ref}$:
\begin{align}
 M_g = \int_{R_{in}}^{R_{out}} \Sigma_g(R) \cdot 2  \pi R \dd R =\int_{R_{in}}^{R_{out}} 2  \pi R \cdot \Sigma_{ref} \left( \frac{R}{R_{ref}}\right)^{-p} \cdot 
 \exp[-(R/R_c)^{(2-p)}] \dd R \\
 \simeq \int_{R_{in}}^{R_{out}} \Sigma_{ref} \left( \frac{R}{R_{ref}}\right)^{-p} \cdot 2  \pi R \dd R =  \begin{cases}
                        2 \pi \Sigma_{ref}R_{ref}^{p} \eval{\frac{R^{-p+2}}{(2-p)}}_{R_{in}}^{R_{out}} , \hspace{5pt} p \neq 2 \\
                        2 \pi \Sigma_{ref}R_{ref}^{2} \eval{\ln R}_{R_{in}}^{R_{out}} , \hspace{5pt} p = 2 \\
\end{cases}
\end{align}

serve quindi inserire gli estremi della simulazione nel DB?


\subsubsection{q index}
\label{qind}
Indice della legge di potenza che descrive il profilo radiale di $c_s$ (e quindi di $H/R$ e $T$)
\begin{equation}
 c_s(R) = c_s(R_{ref}) \left(\frac{R}{R_{ref}} \right)^{-q}
\end{equation}



\section{Workflow simulazioni}


\begin{tikzpicture}

\tikzstyle{block} = [rectangle, rounded corners, minimum width=3cm, minimum height=3cm,text centered, draw=black, fill=gray!30]
\tikzstyle{io} = [rectangle, minimum width=3cm, minimum height=1cm,text centered, align=left, font=\scriptsize]
\node (phantom) [block] {PHANTOM};
\node (mcfost) [block, below of=phantom, yshift=-3cm] {MCFOST};
\node (pymcfost) [block, below of=mcfost, yshift=-3cm] {PYMCFOST};
\node (iph) [io, right of=phantom, xshift=3cm, yshift=0.5cm] {$\leftarrow$ *.setup\\$\leftarrow$ *.in \\
    $\rightarrow$ *.ev \\
    $\rightarrow$ *.discparams \\

};
\node (iph2) [io, below of=iph, yshift=-0.5cm] 
{
    $    \rightarrow
    \begin{cases}
     *\_00001 \\
     *\_00002 \\
     ... \\
    \end{cases}$
};
\coordinate [right of=mcfost, yshift=1.3cm, xshift=2cm] (fn1) ;
\coordinate  [right of=iph2, xshift=1cm] (fn2);
\draw [->] (iph2) -- (fn2) |- (fn1);
\draw [->] (phantom) -- (mcfost);
\draw [->] (mcfost) -- (pymcfost);

\node (imcf) [io, right of=mcfost, xshift=2.6cm, yshift=0.7cm]{ $\leftarrow$ *.para};
\node (uuoutmcf) [io, below of=imcf, xshift=1.5cm, yshift=0.3cm]{ $\rightleftarrows$ data\_th/Temperature.fits.gz};
\node (outmcf) [io, below of=uuoutmcf, xshift=-0.55cm, yshift=0.3cm]{ $\rightarrow$ data\_***/RT.fits.gz};

\coordinate [right of=pymcfost, yshift=1.3cm, xshift=2cm] (fn3) ;
\coordinate  [right of=outmcf, xshift=1cm] (fn4);
\draw[->] (outmcf) -- (fn4) |- (fn3);
\node (ipymc) [io, below of=fn3, xshift=1.28cm, yshift=0.1cm] {$\leftarrow$ bpa, bmaj, bmin};
\end{tikzpicture}

\subsection{Phantom}

Programma realizzato in fortran per simulazioni idrodinamiche mediante il metodo SPH (Smoothed-particle hydrodynamics). Per le simulazioni di dischi costituiti da gas e polvere i passaggi da seguire sono i seguenti:

Si utilizza lo script fornito con phantom per la generazione del makefile specificando il valore della variabile \lstinline{SETUP}, da porre uguale a \lstinline{dustydisc} per simulare dischi costituiti da gas e polvere solida
\begin{lstlisting}
~/sim_dir $ ~/phantom/scripts/writemake.sh dustydisc > Makefile       
\end{lstlisting}

Si utilizza il makefile così creato per generare \lstinline{phantom} e \lstinline{phantomsetup} specificando con il valore della variabile \lstinline{SYSTEM} il compilatore fortran da utilizzare

\begin{lstlisting}
~/sim_dir $ make SYSTEM=gfortran setup
~/sim_dir $ make SYSTEM=gfortran 
\end{lstlisting}

Si può quindi utilizzare phantomsetup per generare il file *.setup tramite l'interfaccia a riga di comando con cui vengono richieste informazioni sul sistema che si vuole simulare.

\begin{lstlisting}
~/sim_dir $ ./phantomsetup sim_name      
\end{lstlisting}

Il file \lstinline{*.setup} deve poi essere modificato manualmente per settare i parametri coi quali eseguire la simulazione. Infine si deve utilizzare nuovamente phantomsetup per generare il file \lstinline{*.in} che viene utilizzato come input per la simulazione.

\begin{lstlisting}
~/sim_dir $ ./phantomsetup sim_name.setup     
\end{lstlisting}

A questo punto si può eseguire la simulazione

\begin{lstlisting}
~/sim_dir $ ./phantom sim_name.in    
\end{lstlisting}

Vengono prodotti i seguenti file:
\begin{enumerate}
 \item[-] \lstinline{*.ev}, file ascii contenenti informazioni globali sul disco in funzione del tempo 
 \item[-] \lstinline{*.discparams}, file ascii contenente i valori delle principali grandezze fisiche del disco con le quali è stata eseguita la simulazione. Vengono generati due file di questo tipo: uno per la componente solida ed uno per la componente gassosa.
 \item[-] file dump con nomi nel formato \lstinline{*\_00001, *\_00002, ...}. Solo alcuni di questi (a seconda di quanto specificato in \lstinline{*.in} sono ``full dumps'' ovvero contengono tutte le informazioni ottenute dalla simulazione e necessarie per, ipoteticamente, farla ripartire da quello stato. Gli altri file dump contengono solo la posizione delle particelle e le smoothing lengths.
\end{enumerate}

\subsection{MCFOST}
Codice che simula il rasferimento radiativo permettendo di ottenere una immagine della radiazione emessa dal disco.
Può essere utilizzato per generare l'emessione nel continuo, l'emissione di linee spettrali specifiche, e l'emissione di radiazione scatterata.

\lstinline{MCFOST} è in grado di leggere direttamente i file dump di phantom. L'istruzione per eseguire una simulazione è la seguente

\begin{lstlisting}
 ~/sim_dir $ mcfost *.para -phantom <dump_file>
\end{lstlisting}

Il file \lstinline{*.para} può essere ottenuto tramite \lstinline{mcfost -get_para}. Costituisce il file di configurazione in cui modificare i parametri prima di eseguire la simulazione. Se si utilizzano i file prodotti da \lstinline{PHANTOM}:
\begin{enumerate}
 \item[-] usare 100 più pacchetti di fotoni rispetto al numero di particelle SPH 
 \item[-] spegnere calcolo della SED (?)
 \item[-] spegnere le simmetrie
\end{enumerate}
Vengono ignorate le seguenti sezioni del file \lstinline{*.para}
\begin{enumerate}
 \item[-] Grid geometry and size
 \item[-] Disk physics
 \item[-] Number of zones
 \item[-] Density structures (except for gas/mass ratio)
\end{enumerate}

Dopo la prima esecuzione viene generato il file \lstinline{data_th/Temperature.fits.gz} che è un archivio gzip contenente un file fits con la temperatura calcolata nel disco. (Come è rappresentata l'informazione? il file fits contiene una sola HDU con un array monodimensionale)
A questo punto si può eseguire nuovamente \lstinline{MCFOST} per determinare l'emissione ad una data lunghezza d'onda 
\begin{lstlisting}
~/sim_dir $ mcfost *.para -phantom <dump_file> -img 870 -casa
\end{lstlisting}
dopo \lstinline{-img} deve essere specificata la lunghezza d'onda a cui si vuole eseguire la simulazione in $\SI{}{\micro\m}$ (in questo esempio 870). L'opzione \lstinline{-casa} può invece essere utilizzata per ottenere un file pronto per \lstinline{CASA}.

Viene generato un file fits (compresso).

\subsection{pymcfost}
Libreria python che fornisce una interfaccia python per l'utilizzo di mcfost. Tra le varie funzioni è interessante per la possibilità di eseguire una convoluzione dell'immagine ottenuta di \lstinline{MCFOST} con beam gaussiani in modo da riprodurre la risoluzione del telescopio.

Utilizzo:
\begin{lstlisting}
import pymcfost as mcfost
mcfost.__version__
%matplotlib inline
import matplotlib 
from matplotlib.patches import Circle

directory = './'
image =  mcfost.Image(directory+"/data_1300/")

img = image.plot(i=0, bpa=159,
                 bmaj=0.14, bmin=0.1, ax=ax[0,1],
                 scale='lin', Jy=True, per_beam=True,
                 no_ylabel=True, no_xlabel=True,
                 plot_stars=False, s=17, limits=[0.8,-0.8,-0.8,0.8])
\end{lstlisting}

il metodo di interesse è quindi il metodo \lstinline{plot()} dell'oggetto \lstinline{Image}. I parametri più importanti di questo metodo sono:

\begin{enumerate}
 \item[-] \lstinline{i}, seleziona quale delle immagini a diverse inclinazioni utilizzare (se nel file file RT.fits sono presenti immagini a diverse inclinazioni)
 \item[-] \lstinline{bpa}, position angle del beam. Se è pari a 0 il beam ha l'asse minore coincidente con l'asse x, crescendo l'asse minore viene ruotato in senso antiorario di un angolo pari a bpa rispetto al semiasse positivo delle ascisse.
\item[-] \lstinline{bmaj}, asse maggiore del beam
\item[-] \lstinline{bmin}, asse minore del beam
\item[-] \lstinline{scale}, scala con cui viene realizzata l'immagine \item[-] \lstinline{Jy=True, per_beam=True}, imposta le unità di misura in $\frac {\text{Jy}}{beam}$. Senza questi parametri le unità di misura sono $\frac {\SI{}{\W\per\m\squared}}{pixel}$. Nota: $1\text{Jy} = \SI{1e-26}{\W\per\m\squared\per\Hz}$
\end{enumerate}

Nota: il parametro \lstinline{s=17} non è più presente nella versione di pymcfost corrente. TODO: capire cos'è e se qualche parametro ora lo sostituisce.

Si può usare
\begin{lstlisting}
 image.writeto(filename='pymcfostsaving.fit')
\end{lstlisting}
per salvare un file fits con i dati dell'immagine ottenuta dall'ultima chiamata di \lstinline{.plot()} e lo stesso header del file utilizzato in input. Non viene aggiornato l'header per cui l'unità di misura è ancora quella di mcfost anche se l'immagine viene prodotta in Jy/beam.       

Realizzo una funzione che utilizza le funzioni precedenti per generare e salvare l'immagine in un file fits. Inoltre modifica l'header del file fits aggiornando o aggiungendo le seguenti informazioni

\begin{center}
\begin{tabular}{l l l}
\toprule
keyword & descrizione & aggiornata o nuova\\
\midrule       
BUNIT & unità di misura dei dati dell'immagine & updated \\ 
CDELT1 & puixel scale x (deg) & kept \\
CDELT2 & puixel scale y (deg) & kept \\
NSTARS & numero di stelle nella simulazione & new \\
TSTAR\# & temperatura della stella (K) & new \\
MSTAR\# & massa della stella in masse solari & new \\
RSTAR\# & raggio della stella in raggi solari & new (?) \\ \\
BMAJ & FWHM maggiore beam & new \\
BMIN & FWHM minore beam & new \\
BPA & position angle del beam & new \\ \\
MGAS & massa della componente gassosa & new\\
MDUST & massa della componente solida & new \\
MPLANET\# & massa dell pianeta in masse di giove & new \\
RORB\# & distanza del pianeta dalla stella & new \\
NPLANETS & numero di pianeti presenti & new \\ \\
DISKPA & position angle del disco & new \\
IMIN & inclinazione minima di quelle nel file RT & new \\
IMAX & inclinazione massima di quelle nel file RT & new \\
NI & numero di inclinazioni nel file RT & new \\
IUSED & indice dell'inclinazione utilizzata nel fits in esame & new \\
DISTANCE & distanza del disco dal telescopio (pc) & new \\
WAVE & lunghezza d'onda a cui è generata l'immagine & kept \\
 \bottomrule
\end{tabular}
\end{center}
\section{Formati}

\subsection{File FITS}
Flexible Image Transport System
Costituito da segmenti chiamati Header/Data Units, ciascuno di questi segmenti è costituito da un header, che contiene varie informazioni sui dati, e un blocco di dati che può essere un array da 1 a 999 dimensionale, una tabella ascii, una tabella binaria (...)

Il file fits di interesse è \lstinline{RT.fits} generato da mcfost
\subsubsection{Primary HDU}

I dati possono essere reucuperati in un array numpy multidimensionale:
\begin{lstlisting}
 data[S,A,I,X,Y]
\end{lstlisting}

\lstinline{S}: può avere diverse dimensioni, questo indice permette di accedere a vari contributi (se calcolati, es. luce scatterata,...) e ai valori dei parametri di Stokes che permettono se combinati di ottenere varie polarizzazioni della luce. La componente non polarizzata è la 0

\lstinline{A}: immagini a diverse inclinazioni azimuthali

\lstinline{I}: immagini a diverse inclinazioni

\lstinline{X,Y}: valori dell'immagine, l'intensità è solitamente salvata in $\frac{\text{W}}{\text{m}\cdot \text{pixel}}$

\section{Data}
\begin{center}
\begin{tabular}{l | c c c c}
\toprule
  & quando & data type & from & py f\\
\midrule       
 $M_P$ & t snapshot & float & image.log & ilf \\ 
 $M_*$ & t snapshot & float & ph:image.log, mc:*.para & pif\\
 $M_g$ & t snapshot & float & image.log & ilf\\
 $M_d$ & t snapshot & float & image.log & ilf \\
 $\frac  HR$ & $R_p$ (?) & float & fixed=0.6 (*.discparams) & *\\
 $R_p$ & t snapshot & float & image.log & ilf \\
 $\Sigma_g(R_p)$ &unperturbed & float & computed  & -\\
 $\alpha_{SS}$ & unperturbed & float & fixed=0.005 & * \\
 $St(R_p)$ & unperturbed &float & computed & - \\
 $s(\lambda)$ &  & float & computed & -\\
 $\rho_g$ & & float & fixed=$\SI{1}{\g\per\cm\cubic}$, from paper & *\\
 $p$ & & float & fixed=1, from paper & *\\
 $q$ & & float & fixed=0.25, from paper & *\\
 $T$ (from phantom) & \\
 $T$ (from mcfost) & \\
 $\kappa$ & \\
 $\beta$ & \\
 $T_*$ & & float & *.para & pif\\
 $b_{maj}$ & &float & pymcfost in & pif\\
 $b_{min}$ & & float & pymcfost in & pif\\
 $b_{pa}$ & & float & pymcfost in & pif\\
 $i$ & & float & *.para & pif\\
 $p.a.$ & & float & *.para & pif\\
 $d$ & & float & *.para & pif\\
 $\lambda$ & & float & RT.fits & ``pif''\\
 $n.orbits$ & & int & name dir\\
 \bottomrule
 26 variabili & \\
\end{tabular}
\end{center}

\section{Organizzazione file ONEDRIVE}

\begin{enumerate}
 \item[-] \lstinline{mcfost_dstau/dstau_Racc} contiene i file dump di phantom
 \item[-] \lstinline{mcfost_dstau/nocasa} contiene cartelle nel formato \lstinline{MP1_time100}, il nome indica la massa del pianeta e il tempo in termini di numero di orbite effettuate dal pianeta al dump in esame. Ogni cartella contiene
 altre cartelle ognuna delle quali si riferisce a simulazioni con diversi gas/dust ($\epsilon$). Formato: \lstinline{gd100}. All'interno di queste cartelle si trovano (che mi possono servire):
 \begin{enumerate}
 \item[-] file \lstinline{.para}, input della simulazione mcfost 
 \item[-] file \lstinline{data_1300/RT.fits.gz} immagine da dare in pasto a pymcfost per ottenere la convoluzione che riproduca la risoluzione del telescopio. Il nome della cartella indica la lunghezza d'onda in micrometri a cui è realizzata l'immagine.
\end{enumerate}
\end{enumerate}


\section{Da valutare}
\begin{enumerate}
 \item[-] architettura (o almeno tipologia) rete neurale: \\ convolutional neural network (?), è trasparente nell'implementazione o l'immagine deve essere processata (es. sezionata, identificare il centro,...) ? (serve?)
 \item[-] dimensione minima necessaria per il database
 \item[-] considerare il tempo?
 \item[-] cosa usare per implementare il database? \\ pandas \\ MLDB \\ Globe \\ MySQL \\ MongoDB \\ file fits con header modificato
 \item[-] variabili adimensionali ? in funzione di cosa ? \\ proposta: database con variabili salvate in SI con possibilità di estrarle in c.u. (da decidere)
 \item[-] distribuzione dati
 \item[-] coerenza temperatura tra mcfost e phantom: provare a fare una verifica. \item[-] cosa hanno in meno i file \lstinline{with_casa} ? passano da circa 10/12 MB a circa 2/3 MB.
 \item[-] posangle beam diverso da posangle disco?
 \item[-] come scegliere dimensione e orientazione beam ? 
\end{enumerate}

\section{TODO}
\begin{enumerate}
 \item[-] riempire tabella subsection 4.1, aggiungere anche dove vanno misurati i parametri
 \item[-] realizzare script che scarica file e info
 \item[-] controllare come sono definiti p.a. e i
\end{enumerate}


\section{Fonti}

\subsection{Paper}
\label{source}
 $^1$\url{https://doi.org/10.1093/mnras/stz913} \\
 $^2$\url{http://dx.doi.org/10.1017/pasa.2018.25}
\subsection{Phantom docs}
\url{https://phantomsph.readthedocs.io/en/latest/index.html}

\subsection{MCFOST docs}
\url{https://ipag.osug.fr/~pintec/mcfost/docs/html/index.html}

\end{document}
